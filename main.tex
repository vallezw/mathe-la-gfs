\documentclass[10pt,aspectratio=169]{beamer}

\usetheme{metropolis}           % Use metropolis theme
\title{GFS Mathematik Lineare Algebra}
\date{20. Januar 2023}
\author{Valentin Zwerschke}
\institute{Königin-Olga-Stift Gymnasium}

\def\titlepage{%
  \usebeamertemplate{title page}%<---
}

\usepackage[german]{babel} % German

\begin{document}
  \maketitle

  \begin{frame}{Gliederung}
	\setbeamertemplate{section in toc}[sections numbered]
	\setbeamertemplate{subsection in toc}[subsections numbered]
	\tableofcontents[hideallsubsections]
  \end{frame}

  \section{Vektoren}
  \subsection{Definition}
  \begin{frame}{Definition}
  \end{frame}


  \subsection{Laenge und Normierung}
  \begin{frame}{Laenge und Normierung}
  \end{frame}

  \subsection{Vektor Addition und Subtraktion}
  \begin{frame}{Vektor Addition und Subtraktion}
  \end{frame}


  \subsection{Skalarprodukt}
  \begin{frame}{Skalarprodukt}
  \end{frame}

  \subsection{Kreuzprodukt}
  \begin{frame}{Kreuzprodukt}
  \end{frame}

  \section{Polygone Netze}
  \subsection{Vertex, Kante, Facette, Oberflaeche}
  \begin{frame}{Vertex, Kante, Facette, Oberflaeche}
  \end{frame}

  \subsection{Flaechennormale und Orientierung}
  \begin{frame}{Flaechennormale und Orientierung}
  \end{frame}


  \subsection{ggf. Gerade und Ebene in Vektorform}
  \begin{frame}{ggf. Gerade und Ebene in Vektorform}
  \end{frame}

  \section{Matrizen}

  \subsection{Defintion und lsg LGS}
  \begin{frame}{Definition und lsg LGS}
  \end{frame}

  \section{Transformation}

  \subsection{Affine Transformation: Skalierung und Rotation}
  \begin{frame}{Affine Transformation: Skalierung und Rotation}
  \end{frame}

  \subsection{Translation, homogene Koordinaten}
  \begin{frame}{Translation, homogene Koordinaten}
  \end{frame}

  \section{Anwendung auf Computer Grafik}


  \subsection{Perspektivische Projektion}
  \begin{frame}{Perspektivische Projektion}
  \end{frame}


  \subsection{Screen Mapping}
  \begin{frame}{Screen Mapping}
  \end{frame}

  \subsection{Abbildung eines Drehbaren Dreiecks im Raum}
  \begin{frame}{Abbildung eines Drehbaren Dreiecks im Raum}
  \end{frame}

\end{document}