\documentclass[10pt,aspectratio=169]{beamer}

\usetheme{metropolis}           % Use metropolis theme
\title{GFS Mathematik Lineare Algebra}
\date{20. Januar 2023}
\author{Valentin Zwerschke}
\institute{Königin-Olga-Stift Gymnasium}

\def\titlepage{%
  \usebeamertemplate{title page}%<---
}


\usepackage{tikz}
\usepackage[german]{babel} % German

\begin{document}
  \maketitle

  \begin{frame}{Gliederung}
	\setbeamertemplate{section in toc}[sections numbered]
	\setbeamertemplate{subsection in toc}[subsections numbered]
	\tableofcontents[hideallsubsections]
  \end{frame}

  \section{Vektoren}
  \subsection{Vektoren}
  \begin{frame}{Vektoren}
    \begin{minipage}{0.65\textwidth}
      \begin{itemize}
        \item Punkte im Raum werden ueber Vektoren beschrieben 
        \item Vektor Koordinaten $v_x$ und $v_y$ (Kartesischem Koordinaten System = Achsen Senkrecht)\\\vspace{0.2cm} 
        \hspace{0.3cm}$\vec{v} = \begin{pmatrix} v_x\\ v_y\end{pmatrix}$
        \vspace{0.2cm}
        \item Laenge (euklidische Norm) bsp. 2D, 3D\\
        \hspace{0.3cm}$||\vec{v}|| =  \sqrt{v_x^2 + v_y^2}$\\
        \hspace{0.3cm}$||\vec{v}|| =  \sqrt{v_x^2 + v_y^2 + v_z^2}$
        \vspace{0.2cm}
        \item Richtung/Normierter Vektor\\
        \hspace{0.25cm}\Large$\frac{\vec{v}}{||\vec{v}||}$
      \end{itemize}  
    \end{minipage}
    \begin{minipage}[c]{0.3\textwidth}
      \begin{tikzpicture}
        \draw[thick,->] (0,0) -- (3.5,0) node[anchor=north west] {x};
        \draw[thick,->] (0,0) -- (0,2.5) node[anchor=south east] {y};
        \draw[thick,->, blue] (0,0) -- node[above] {$\vec{v}$} (3,2) node[anchor=north east] {};
        \draw[densely dotted]  (3,0) node[below] {$v_x$} -- (3,2);
        \draw[densely dotted]  (0,2) node[left] {$v_y$} -- (3,2);
        %\node [black] at (3,2) {\textbullet};
      \end{tikzpicture}
    \end{minipage}
  \end{frame}

  \subsection{Rechnen mit Vektoren}
  \begin{frame}{Rechnen mit Vektoren}
    \vspace{0.2cm}
    \begin{minipage}{6.5cm}
      \textbf{Addition}
      \begin{itemize}
        \item Jeweils Summe der Koordinaten\\
        \vspace{0.2cm}
        \hspace{0.2cm}
        $\vec{u} + \vec{v} = \begin{pmatrix}u_x + v_x\\ u_y + v_y\end{pmatrix}$
        \vspace{-0.5cm}
      \end{itemize}
      \hspace{0.25cm}
      \begin{tikzpicture}
        \draw[thick,->] (0,0) -- (3.5,0) node[anchor=north west] {x};
        \draw[thick,->] (0,0) -- (0,2.5) node[anchor=south east] {y};

        \draw[thick,->, blue] (0,0) -- node[above] {$\vec{u}$} (3,1);
        \draw[thick,->, blue] (3,1) -- node[above] {$\vec{v}$} (1.5,2);
        \draw[thick,->, red]   (0,0) -- node[above] {\hspace{1.4cm}\small$\vec{u} + \vec{v}$} (1.5,2);
      \end{tikzpicture}
    \end{minipage}
    \begin{minipage}{6.5cm}
      \textbf{Subtraktion}
      \begin{itemize}
        \item Jeweils Differenz der Koordinaten\\
        \vspace{0.2cm}
        \hspace{0.2cm}
        $\vec{v} - \vec{u} = \begin{pmatrix}v_x - u_x\\ v_y - u_y\end{pmatrix}$
        \vspace{-0.5cm}
      \end{itemize}
      \hspace{0.25cm}
      \begin{tikzpicture}
        \draw[thick,->] (0,0) -- (3.5,0) node[anchor=north west] {x};
        \draw[thick,->] (0,0) -- (0,2.5) node[anchor=south east] {y};

        \draw[thick,->, blue] (0,0) -- node[above] {$\vec{u}$} (3,1);
        \draw[thick,->, red] (3,1) -- node[above] {\hspace{0.5cm}\small$\vec{v} - \vec{u}$} (1.5,2);
        \draw[thick,->, blue]   (0,0) -- node[above] {$\vec{v}$} (1.5,2);
      \end{tikzpicture}
    \end{minipage}
    \textbf{Skalierung}
    \begin{center}
      $\lambda \vec{v} = \begin{pmatrix}\lambda v_x\\ \lambda v_y\end{pmatrix}$
    \end{center}
  \end{frame}


  \subsection{Skalarprodukt}
  \begin{frame}{Skalarprodukt}
  \end{frame}

  \subsection{Kreuzprodukt}
  \begin{frame}{Kreuzprodukt}
  \end{frame}

  \section{Polygone Netze}
  \subsection{Vertex, Kante, Facette, Oberflaeche}
  \begin{frame}{Vertex, Kante, Facette, Oberflaeche}
  \end{frame}

  \subsection{Flaechennormale und Orientierung}
  \begin{frame}{Flaechennormale und Orientierung}
  \end{frame}


  \subsection{ggf. Gerade und Ebene in Vektorform}
  \begin{frame}{ggf. Gerade und Ebene in Vektorform}
  \end{frame}

  \section{Matrizen}

  \subsection{Defintion und lsg LGS}
  \begin{frame}{Definition und lsg LGS}
  \end{frame}

  \section{Transformation}

  \subsection{Affine Transformation: Skalierung und Rotation}
  \begin{frame}{Affine Transformation: Skalierung und Rotation}
  \end{frame}

  \subsection{Translation, homogene Koordinaten}
  \begin{frame}{Translation, homogene Koordinaten}
  \end{frame}

  \section{Anwendung auf Computer Grafik}


  \subsection{Perspektivische Projektion}
  \begin{frame}{Perspektivische Projektion}
  \end{frame}


  \subsection{Screen Mapping}
  \begin{frame}{Screen Mapping}
  \end{frame}

  \subsection{Abbildung eines Drehbaren Dreiecks im Raum}
  \begin{frame}{Abbildung eines Drehbaren Dreiecks im Raum}
  \end{frame}

\end{document}